%%%%%%%%%%%%%%%%%
% This is an example CV created using altacv.cls (v1.1.4, 27 July 2018) written by
% LianTze Lim (liantze@gmail.com), based on the
% Cv created by BusinessInsider at http://www.businessinsider.my/a-sample-resume-for-marissa-mayer-2016-7/?r=US&IR=T
%
%% It may be distributed and/or modified under the
%% conditions of the LaTeX Project Public License, either version 1.3
%% of this license or (at your option) any later version.
%% The latest version of this license is in
%%    http://www.latex-project.org/lppl.txt
%% and version 1.3 or later is part of all distributions of LaTeX
%% version 2003/12/01 or later.
%%%%%%%%%%%%%%%%

%% If you want to use \orcid or the
%% academicons icons, add "academicons"
%% to the \documentclass options.
%% Then compile with XeLaTeX or LuaLaTeX.
% \documentclass[10pt,a4paper,academicons]{altacv}

%% Use the "normalphoto" option if you want a normal photo instead of cropped to a circle
% \documentclass[10pt,a4paper,normalphoto]{altacv}

\documentclass[10pt,a4paper]{altacv}

%% AltaCV uses the fontawesome and academicon fonts
%% and packages.
%% See texdoc.net/pkg/fontawecome and http://texdoc.net/pkg/academicons for full list of symbols.
%% When using the "academicons" option,
%% Compile with LuaLaTeX for best results. If you
%% want to use XeLaTeX, you may need to install
%% Academicons.ttf in your operating system's font %% folder.


% Change the page layout if you need to
\geometry{left=1cm,right=9cm,marginparwidth=6.8cm,marginparsep=1.2cm,top=1cm,bottom=1cm}

% Change the font if you want to.

% If using pdflatex:
\usepackage[utf8]{inputenc}
\usepackage{crimson}
\usepackage[T1]{fontenc}
% \usepackage[default]{lato}

% If using xelatex or lualatex:
% \setmainfont{Lato}
% Change the colours if you want to
\definecolor{VividPurple}{HTML}{1E4889}
\definecolor{SlateGrey}{HTML}{2E2E2E}
\definecolor{LightGrey}{HTML}{666666}
\colorlet{heading}{VividPurple}
\colorlet{accent}{VividPurple}
\colorlet{emphasis}{SlateGrey}
\colorlet{body}{LightGrey}

% Change the bullets for itemize and rating marker
% for \cvskill if you want to
\renewcommand{\itemmarker}{{\small\textbullet}}
\renewcommand{\ratingmarker}{\faCircle}

%% sample.bib contains your publications
\addbibresource{sample.bib}

\begin{document}
\name{Emmanuel Ruaud}
\tagline{D\'eveloppeur logiciel}

\photo{2.5cm}{pic}
\personalinfo{%
  % Not all of these are required!
  % You can add your own with \printinfo{symbol}{detail}
  \phone{06.67.91.84.41}
  \email{em.ruaud@gmail.com}
  \location{Villeurbanne, France}
  \printinfo{\faLinkedin}{eruaud}
  \printinfo{\faGithub}{o-reo}
}

%% Make the header extend all the way to the right, if you want.
\begin{fullwidth}
\makecvheader
\end{fullwidth}

%% Depending on your tastes, you may want to make fonts of itemize environments slightly smaller
\AtBeginEnvironment{itemize}{\small}

%% Provide the file name containing the sidebar contents as an optional parameter to \cvsection.
%% You can always just use \marginpar{...} if you do
%% not need to align the top of the contents to any
%% \cvsection title in the "main" bar.

\cvsection[page1sidebar]{Exp\'eriences}

\cvevent{D\'eveloppeur C++ et Python}{Freelance}{Juil 2019 -- Aujourd'hui}{Lyon}
Interventions ponctuelles en d\'eveloppement logiciel pour les soci\'et\'es Snatch et Substances actives.
Secteur m\'edical et marketing publicitaire.\linebreak
\divider

\cvevent{Formateur C++}{Campus Ynov}{Novembre 2019 -- Aujourd'hui}{Lyon}
Formation sur le langage C++ \`a destination des L3 en d\'eveloppement logiciel lors d'un module de 24h de formation th\'eorique et pratique, suivi de 32h d'encadrement de projet.\linebreak
\divider

\cvevent{D\'eveloppeur C++}{Teazit}{Jan 2019 -- Juil 2019}{Lyon}
D\'eveloppement de la V2 du logiciel embarqu\'e de cam\'eras interconnect\'ees \`a destination de l'\'evenementiel.\linebreak
\divider

\cvevent{Ing\'enieur acousticien}{QCS Services}{F\'ev 2014 -- Juin 2017}{Noisy-le-Grand}
Mesures et \'etudes acoustiques dans le b\^atiment. Pilote d\'eveloppement de logiciels en VBA et Python.\linebreak
\divider

\cvevent{Ing\'enieur validation}{Free Field Technologies}{Sept 2012 -- F\'ev 2013}{Louvain-la-neuve, Belgique}
Optimisation d'une m\'ethode hybride PML/IFEM applicable au logiciel ACTRAN.\linebreak
Validation de la m\'ethode en collaboration avec l'\'equipe d\'eveloppement. Python, serveurs de calcul Linux, logiciels de maillage 3D.
Int\'eractions avec les secteurs de l'a\'eronautique et de l'automobile.\linebreak
\divider

\cvevent{D\'eveloppeur logiciel}{Harman international}{Juil 2011 --  Sept 2011}{Ch\^ateau-du-Loir}
D\'eveloppement d'un logiciel acoustique de calibration de syst\`emes audio int\'egr\'es. Traduction de C\# \`a
Matlab et ajout de fonctionnalit\'es.\linebreak

\clearpage

\cvsection[page2sidebar]{Formations}

\cvevent{D\'eveloppeur d'application}{\'Ecole 42}{Nov 2017 -- Aujourd'hui}{Lyon}
Technologies de l'information.\linebreak
Ecole utilisant la p\'edagogie pair \`a pair avec un apprentissage \`a la carte le long de projets concrets, professionnalisants, seuls ou en \'equipe.\linebreak
\divider

\cvevent{Master 2 en Acoustique}{Technical University of Denmark}{Décembre 2012 -- Juin 2013}{Copenhague}
Acoustique fondamentale, architecturale et psychoacoustique.\linebreak
Programme Erasmus de 6 mois.\linebreak
\divider

\cvevent{Ing\'enieur G\'en\'eraliste}{ENSIM}{Sept 2009 -- Juin 2012}{Le Mans}
Option Acoustique et Vibrations.\linebreak
Cursus g\'en\'eraliste avec notamment des modules de programmation en C++, \'ethique, m\'ecanique des fluides, \'electronique et gestion de projet.
\divider

\cvevent{Classes pr\'eparatoires}{Universit\'e Blaise Pascal}{Sept 2007 -- Juin 2009}{Clermont-Ferrand}
Classes pr\'eparatoires aux grandes \'ecoles option Physique et Sciences de l'Ing\'enieur (PSI).

\clearpage
\end{document}
